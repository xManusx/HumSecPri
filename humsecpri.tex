\documentclass[a4paper,12pt]{scrartcl}
\usepackage[utf8]{inputenc}
\usepackage[T1]{fontenc}
\usepackage{lmodern}
\usepackage{amsmath}
\usepackage{amssymb}
\usepackage{makeidx}
\makeindex{}
\usepackage{hyperref}
\usepackage{csquotes}
\usepackage{graphicx}

\usepackage[backend=biber,style=numeric]{biblatex}
\addbibresource{bibliography.bib}


%\renewcommand{\thesection}{Sitzung \Roman{section}: }
%\usepackage{titlesec}
%\titlelabel{Sitzung \thesection{} --- }
\title{Human Factors in Security and Privacy}
\author{\href{mailto:magnus.berendes@fau.de}{magnus.berendes@fau.de}}
\date{\today}
\begin{document}
\maketitle

\vfill
\begin{centering}
	Corrections, annotations and pull requests appreciated
	\\
\end{centering}


\newpage

\tableofcontents
\newpage
\printindex
\printbibliography
\newpage

\section{Introduction}
\begin{itemize}
	\item
		What is security?\index{Security}
		\begin{itemize}
			\item
				Protect the right thing in a right way (Anderson)
			\item
				Risk management: Trade-off between risk of attack and cost of protection
		\end{itemize}
	\item
		Defining Security:
		\begin{itemize}
			\item
				Goals: \textit{what} to protect

				Security properties of assets: CIA\index{CIA} (Confidentiality, Integrity, Availability)
			\item
				Threats: \textit{against what/whom} to protect
			\item
				Means: \textit{how} to protect

				Safeguards (attack prevention) and Countermeasures (detection and response)
		\end{itemize}
	\item
		Human Factors as Protection Means:
		\begin{itemize}
			\item
				Security management processes: policies, rules, decisions
			\item
				Usable security
			\item
				User education and training (apply with care!)
		\end{itemize}

	\item
		Security-questions to ask:
		\begin{itemize}
			\item
				System:
				\begin{itemize}
					\item
						Technical structure/organization
					\item
						Stakeholders: user types, service providers
					\item
						Assets: what should be protected?
				\end{itemize}
			\item
				Security goals? CIA?
			\item
				Threats/Attackers?
				\begin{itemize}
					\item
						Which threats are possible, probable, incur high losses?
					\item
						Incentives of the attackers?
					\item
						Resources/capabilities of the attackers?
				\end{itemize}
			\item
				Does the security measure protect against these attacks?
			\item
				If yes, is the protection \textit{effective}?
				\begin{itemize}
					\item
						Cost of protection vs. attack risk
					\item
						Are users capable of the required behavior?
				\end{itemize}
			\item
				Is it \textit{efficient}?
				\begin{itemize}
					\item
						Cost of protection (money, time, user effort (!) ) vs. attack risk
				\end{itemize}
		\end{itemize}
	\item

		\begin{figure}[ht]
			\centering
			\includegraphics[
			width=0.8\textwidth]{security_process.jpg}
		\end{figure}

	\item
		Many S\&P measures offer bad trade-offs to the user: not much (subjective) gain of security, but a hassle
	\item
		\textit{Authorization}: \index{Authorization}
		\begin{itemize}
			\item
				Identification (e.g. login name)
			\item
				Plus authentication (e.g. password)
		\end{itemize}
\end{itemize}



\section{Passwords}
\begin{itemize}
	\item
		Security goals:
		\begin{itemize}
			\item
				Keep the bad guys out
			\item
				Don't lock me out!
		\end{itemize}
	\item
		Traditional password advice: strong (hard to guess), don't use it anywhere else, don't write it down, don't share it, change it often
	\item
		Do people follow this advice? No!
\end{itemize}
\subsection{Users are not the enemy}
\begin{itemize}
	\item
		\fullcite{usersenemy}
	\item
		Asked users about their password construction, frequency of use of different passwords, password recall
	\item
		Password diaries: write down all \enquote{happenings} around the password based authentication
	\item
		Findings:
		\begin{itemize}
			\item
				Users do not comply with policies
			\item
				but are not being wicked or stupid:
			\item
				They lack the right mental models of threats
			\item
				Password mechanisms lack user-centered design: not aligned with human memory capacities, not aligned with workflows (working together, delegating)
		\end{itemize}

		$\Rightarrow$ People chose weak passwords, reuse passwords, write them down, share them
	\item
		Mental Model \index{Mental Model}:
		\begin{itemize}
			\item
				Internal symbolic representation of external reality
			\item
				E.g. password=key model (how do i know if the website I'm logging into is real or fake? The key fits!)
			\item
				\enquote{Hackers cannot know the name of my cat, therefore that's a safe password}
		\end{itemize}
\end{itemize}

\subsection{A large-scale study of web password habits}
\begin{itemize}
	\item
		\fullcite{florenciolarge}
	\item
		544960 web clients, 3 months
	\item
		Windows Live Toolbar component with opt-in
	\item
		Report findings: websites, password compostion statistics (passwords were not transmitted), reuse, number of login attempts, time since last login
	\item
		Numnber of accounts per user: 25
	\item
		Each password shared across 4 different sites (avg)
	\item
		Passwords used per day: 8
	\item
		Mostly only lowercase letters \textit{unless forced to do otherwise}
\end{itemize}

\subsection{Passwords \& Human capabilities}
\begin{itemize}
	\item
		Why password logins fail:
		\begin{itemize}
			\item
				Failure to recall pssword

				Limited capacity of memory, decay over time, only unaided recall (no cues), items are non-meaningful, passwords are confused (items are similar and compete in memory), remember old password
			\item
				Typing errors

				no feedback on failure, needs to be 100\% correct
			\item
				Incorrect User ID
			\item
				User ID $\leftrightarrow$ password confusion
			\item
				Lock out after 3 attemps:

				Increasing to 10 tries decreases resets by 45\%
		\end{itemize}
	\item
		\fullcite{inglesanttrue}
		\begin{itemize}
			\item
				Data collection:  password diaries and interviews at two companies
			\item
				Conclusion very similar to \enquote{Users are not the enemy} from 1999 --- no positive changes after 10 years
		\end{itemize}
	\item
		\fullcite{florenciosecurity}
		\begin{itemize}
			\item
				Policies concentrate on forcing users to produce \enquote{strong} passwords
			\item
				Analysed \index{Minimal strength of password policies} minimal strength of password policies over 75 websites
			\item
				Minimal strength of a password policy: $N_{min} \log_2 C$

				$N_{min}$: minimal length, $C$: Character Space
			\item
				Analyses strength of the policy, not the password (strength of \enquote{password} is roughly the same as of \enquote{p4\$\$w0rd}, policy of the latter is much stronger)
		\end{itemize}
	\item
		Strength of password policies in the wild: FAU IdM much stronger than the rest ($48 > 20-27$)

		Sites with advertising often much weaker policies $\Rightarrow$ want user to sign up, password policies prevent that!
\end{itemize}
\subsection{Attacks on passwords}
\begin{itemize}
	\item
		Client: local attacks (shoulder surfing, post-its), social engineering, keyloggers
	\item
		Network:
		eavesdropping, MitM
	\item
		Server frontend: Online guessing (brute-force/dictionary) --- breadth-first (target all accounts), depth-first (target particular account), stealthy attacks (distributed in time)
	\item
		Server backend: offline guessing (prerequisite: obtain password database)
	\item[$\Rightarrow$] strong password policies protect against offline guessing attacks --- offline guessing attacks only feasible when password database is leaked, leak is not detected and passwords are salted and hashed
	\item
		Lookup tables/Rainbow tables: TODO
		\begin{itemize}
			\item
				Pre-compute hashes of passwords and look them up
			\item
				Password spaces quite big (10 ULNS policy needs more than $2^{65}$ hashes $ \mathrel{\widehat{=}} 1000$ Exabyte)
		\end{itemize}
	\item
		Online-offline chasm:
		\begin{itemize}
			\item
				Strength of passwords against online guessing: $10^4 - 10^6$ guesses
			\item
				Strength of passwords against offline guessing: $10^{14} - 10^{20}$ guesses
		\end{itemize}
		
	\item
		Summary to strong passwords:

		Strong passwords\dots
		\begin{itemize}
			\item
				reduce risk of offline guessing (iff database is hashed and salted)
			\item
				reduce risk of online guessing (but with lock-out and monitoring much weaker passwords are sufficient)
			\item
				might reduce risk against shoulder surfing and insider guessing
			\item
				don't protect against: malware, phishing, network attacks
			\item
				might increase risk of password reuse and writing down
		\end{itemize}
\end{itemize}

\subsection{Password Reuse}
\begin{itemize}
	\item
		\fullcite{tangledpasswordreuse}
		\begin{itemize}
			\item
				Study of password reuse with leaked passwords from eleven web sites
			\item
				Lots of reuses: 43\% identical, 19\% substrings
			\item
				Often tricks to transform passwords to comply with policies
		\end{itemize}
	\item
		\fullcite{florencio2014password}
		\begin{itemize}
			\item
				Lots of effort to remember which password belongs to which account \\
				($\log_2(N!)$, for N unique and random passwords)
			\item
				Unique passwords is not the optimal solution $\Rightarrow$ optimally, people should reuse passwords, group by account value and probability of compromise
		\end{itemize}
	\item
		Password Sharing
		\begin{itemize}
			\item
				Reasons for password sharing: practical needs (if something happens to me, delegation, disabilities), social norms (as a sign of trust, to show you're not paranoid, I'm a normal, nice, helpful person)
		\end{itemize}
	\item
		Password Change
		\begin{itemize}
			\item
				\fullcite{zhang2010security}
			\item
				Access to expired passwords at university, assuming knowledge of at least one password, guessed 17\% in 5 tries or lower (online), 41\% in under 3 seconds of offline attacking
		\end{itemize}
	\item
		Password Management 101
		\begin{itemize}
			\item
				Providers:
				\begin{itemize}
					\item
						Reasonable password policies and SSO
					\item
						Protection from offline guessing (hashed and salted, detect offline attacks e.g. with decoy accounts)
					\item
						Protection from online guessing (lock out after some failures, detect stealth attacks)
					\item
						Be prepared (usable and secure account recovery mechanisms, actionable strategies for the case of mass breaches)
				\end{itemize}
			\item
				Users
				\begin{itemize}
					\item
						Group passwords by account importance and attack probability (non important passwords can be reused; are users capable of distinguishing non/important accounts? Of determining attack probability?)
					\item  
						Don't reuse passwords on really important accounts
					\item
						Use aids for password management (password managers, secure writing down)
				\end{itemize}
		\end{itemize}
	\item
		Alternatives to passwords: graphical authentication, biometrics, tokens, 2FA
\end{itemize}

\section{Usable Security}
\begin{itemize}
	\item
		Usability: the extent to which a product can be used by specified users to achieve specified goals with effectiveness, efficiency and satisfaction in a specified context of use.
		\begin{itemize}
			\item
				Effectifeness: accuracy and completeness
			\item
				Efficiency: resources expended in relation to accuracy and completeness
			\item
				Satisafction: freedom from discomfort, and positive attitude to use of product
			\item
				context of use: characteristics of the users, tasks and environment
			\item
				Goal: intended outcome
			\item
				Task: activities required to achieve a goal
		\end{itemize}
	\item
		Usability Intervention at IBM 1989
		\begin{itemize}
			\item
				Field study: user group, environment, task (not authenticate, but \textit{do these 4 transactions})
			\item
				3 iterations (from mock-up to test code)
			\item
				100\% return on invested money, through time  saved while authenticating
		\end{itemize}
	\item
		\fullcite{johnnyencrypt}
		\begin{itemize}
			\item
				Security software is usable if people who are expected to use it:
				\begin{enumerate}
					\item
						Are reliably made aware of the security tasks they need to perform
					\item
						are able to figure out how to successfully perform those tasks
					\item
						don't make dangerous errors
					\item
						are sufficiently comfortable with the interface to continue using it
				\end{enumerate}
			\item
				experienced email users, no knowledge in cryptography in a lab, 12 participants
		\end{itemize}
	\item
		\fullcite{johnny2015}
		\begin{itemize}
			\item
				10 pairs of students, exchange tax return information with Mailvelope
			\item
				Results desastrous, only 1 pair completed the task (one of them knew about public key crypto before)
			\item
				All without attacker interference!
		\end{itemize}
	\item
		\fullcite{johnny2}
		\begin{itemize}
			\item
				Send sensitive data with Outlook and S/MIME, certificates are pre-loaded into mail clients of participants
			\item
				Participants don't know they are going to be attacked
			\item
				New Key attack: \enquote{computer problems}, send schedule to new address, new address has unmatching key from old address
			\item
				New Identity attack: \enquote{working from home}, new hotmail email with a new public key and misspelled email address (sara instead of sarah)
			\item
				Unsigned message attack: attacker impersonates manager being annoyed, known address, unsigned mail. Send schedule to team members NOW as they really need it
			\item
				3 groups: unmodified outlook, outlook with color coding (for secure and unsecure messages) and color coding + briefing about meaning of colors
			\item
				Conclusion: usable email clients help \textit{sometimes}: resisting attacks better with color/color+briefing, clicked encrypt is worse than with unmodified

				Some users did not trust \textit{hotmail}, ability to encrypt a message for the sender taken for indicator that the sender is genuine
		\end{itemize}
	\item
		Why is usable security hard?

		Usability: the extent to which a product can be used by specified users to achieve specified goals with effectiveness, efficiency and satisafaction in a specified context of use
		\begin{itemize}
			\item
				Product: Security is often integrated/part of product
			\item
				User Goals: Security = secondary goal
			\item
				Effectiveness: how can security be noticed? Often invisible
			\item
				Efficiency: additional time \& user effort
			\item
				Satisfaction: can people be satisfied with something that demands additional effort? Maybe better: low dissatisfaction?
			\item
				context of use: How to take attacks into account?
		\end{itemize}
	\item
		Benenson defintion of usable security:
		\begin{itemize}
			\item
				Usability of a security mechanism or policy
			\item
				is the extent to which the machnism or policy can be used by
			\item
				specified users
			\item
				to achieve the specified security goal (some of these goals being invisible or secondary goals to the users)
			\item
				and the specified primary goals
			\item
				with effectiveness, efficiency and satisfaction (or at least with low dissatisfaction)
			\item
				during the execution of the specified primary user tasks
			\item
				in a specified context of use
			\item
				including the specified attacks
		\end{itemize}
\end{itemize}
\subsection{Beyond Usability}
\begin{itemize}
	\item
		Is usability \& security a trade-off? long random passwords, single password per account, encrypted email, \dots
		\begin{itemize}
			\item
				Security versus Usability: security restricts/prohibits, usability enables
			\item
				Security aligned with Usability: Usability enables actions with desirable results, security restricts actions with undesirable results
		\end{itemize}
	\item
		Can't just add security to finished product $\Rightarrow$ usability will suffer
	\item
		Approaches to usable security (Apply in this order!)
		\begin{itemize}
			\item
				Invisible security: Make it just work, automated and invisible
			\item
				Understandable security: Make it visible and intuitive
			\item
				Train the users: effort should match benefit
		\end{itemize}
	\item
		Alternatives to passwords for web authentication:
		\begin{itemize}
			\item
				Criteria: Usability; Security/Privacy; Deployability
			\item
				Evaluated 35 alternative techniques: graphical passwords/PINs, Biometrics, Tokens, 2FA
			\item
				Results: many techniques would be better for security and privacy buuuut none has better deployability (for Web)
		\end{itemize}
	\item
		Usability criteria for web authentication:
		\begin{itemize}
			\item
				Memorywise (quasi) effortless
			\item
				Scalable for users (many accounts)
			\item
				Nothing to carry (or something the user carries anyway)
			\item
				Physically effortless (beyond pressing a button)
			\item
				Easy to learn
			\item
				Efficient to use (time)
			\item
				Infrequent errors
			\item
				Easy recovery from loss
		\end{itemize}

		Security and privacy criteria: resilience to
		\begin{enumerate}
			\item
				Physical observation
			\item
				targeted impersonation (having personal info about user does not help)
			\item
				throttled guessing (guessing attempts are constrained by service provider)
			\item
				Unthrottled guessing (attacker constrained only by computational resources)
			\item
				Internal observation (intercept users input)
			\item
				Leaks at other service providers
			\item
				Phishing
			\item
				theft of physical token (or at least pin protection)
			\item
				No trusted third party
			\item
				Explicit consent (authentication cannot be startet without user knowledge or consent)
			\item
				Unlinkable (colluding service providers cannot track the users from their authentication data)
			\item
				Privacy-protecting (does not reveal additional information about the user to the service provider / attacker
		\end{enumerate}

		Deployability Criteria
		\begin{enumerate}
			\item
				Accessible
			\item
				Low cost per user
			\item
				Server-compatible
			\item
				Browser-compatible
			\item
				Mature (User testing, open source implementations, real-world deployments)
			\item
				Non-proprietary
		\end{enumerate}
	\item
		Why doesn't Jane protect her Privacy? Renaud et.al.
		\begin{itemize}
			\item
				Qualitative study
		\end{itemize}
\end{itemize}

\section{Cyberfraud}
\begin{itemize}
	\item
		Fraud: crime of intentional deception
	\item
		Phishing: gain secret information (often via email)
	\item
		Scam: gain money (f2f/long distance: letters, emails)
	\item
		Social engineering: manipulate people. Gain information, make them perform action
\end{itemize}
\subsection{Phishing}
\begin{itemize}
	\item
		\fullcite{socialphishing}
		\begin{itemize}
			\item
				Crawl online social network sites, send phishing email with a link 
				and \enquote{hew, check this out!}
			\item
				Social phishing group: from an OSN friend, control group: \enquote{Normal} phishing
			\item
				Ask to log into website with phishy link, forward credentials to real website
			\item
				Results:
				\begin{itemize}
					\item
						Control group: 16\% phished, Social phishing group: 72\% phished
					\item
						Big ethics outcry: students were not aware they participated in the experiment, only prof, Insitutional Review Board and 2 IT guys knew
					\item
						Students who were phished complained (university took their trust and threw it out of the window) and students whose emails were used to phish, IT Department
				\end{itemize}
		\end{itemize}
	\item
		\fullcite{jakobsson2008}
		\begin{itemize}
			\item
				Harm done? Credentials obtained? Other forms?
			\item
				Informed consent before the experiment obtained?
			\item
				Debriefing: yes or no? What causes more harm?
		\end{itemize}
	\item
		\fullcite{downs2006decision}
		\begin{itemize}
			\item
				How do people decide over trustworthiness of emails?
			\item
				20 participants, role playing, didn't know experiment was about phishing or scam
			\item
				Results 
				\begin{enumerate}
					\item
						This email seems to be for me
					\item
						It's normal to hear from companies you do business with
					\item
						Reputable companies will send emails
				\end{enumerate}

		\end{itemize}
	\item
		\fullcite{2006phishing}
		\begin{itemize}
			\item
				20 e-commerce or banking websites (7 real, other collected or constructed)
			\item
				22 participants, distinguish which sites are fake
			\item
				Good phishing worked on 90\% of users
			\item
				Certificate warnings and security indicators were ignored a lot
			\item
				Independent of education, age, sex, computer experience
			\item
				Decision strategies: Security indicators in website content only, Content and domain name only, Content and address, plus \textsc{https}, All the above plus padlock icon, All the above plus certificates (kind of sorted by strength)
		\end{itemize}
	\item
		\fullcite{2015phishing}
		\begin{itemize}
			\item
				Lab experiment, identify phishing websites
			\item
				Participants average success rate of 53\% for identifying phishing websites $\Rightarrow$ guessing
			\item
				While making decisions, only 6\% of time looking at security indicators, 85\% looking at contents of webpage
			\item
				Technical proficency does not correlate with improved detection scores
		\end{itemize}
	\item
		Spear-Phishing: Trageting a group / an individual (by name, spoof trusted sender, plausible content)
	\item
		Studies by Benenson et.\ al.
		\begin{itemize}
			\item
				Questions:
				\begin{itemize}
					\item
						Email vs. Facebook, differences in attack effectiveness
					\item
						Reasons for clicking and not clicking?
				\end{itemize}
			\item
				Consent \& Deception: participants recruited for survey about \enquote{online behavior} + Debriefing about real aim of study
			\item
				Study 1 (with name): more effective via email
			\item
				Study 2 (without name): more effective via facebook
			\item
				Nach (nicht-)klicken Nachricht mit Bitte warum (nicht) geklickt wurde
			\item
				Analysemethode der Antworten: 
				\begin{itemize}
					\item
						Qualitative Content Analysis
					\item
						Iteratice Coding
					\item
						2 Raters and Cohens Kappa\footnote{How much the correlation between the 2 ratings differs from random}
				\end{itemize}
			\item
				Reasons for clicking:
				\begin{itemize}
					\item
						Curiosity / interest (34\%)
					\item
						Plausible content, fits expectations (27\%)
					\item
						Investigation (17\%)
					\item
						Technical context, Fear, Automatic reaction
				\end{itemize}
				
			\item
				What makes people click:
				\begin{itemize}
					\item
						Personalization
					\item
						Plausible content and context
					\item
						Emotions: curiosity, interest, fear
					\item
						Risk underestimation
					\item
						automatic reactions
				\end{itemize}
			\item
				reasons for not clicking:
				\begin{itemize}
					\item
						Unknown sender, suspicion of fraud, situation context, life context, rule of conduct, privacy
				\end{itemize}
		\end{itemize}
	\item
		Measures against Phishing
		\begin{itemize}
			\item
				Invisible Security
				\begin{itemize}
					\item
						Detection an blocking on provider side: takes up to 12 hours, lots of clicks in the first 2 hours, links  mostly work for less than 24 hours
					\item
						Automatic detection on user side: too many false negatives and false positives
				\end{itemize}
			\item
				Understandable security:
				\begin{itemize}
					\item
						Indicators / warnings
					\item
						Poor accuracy of automatic detection
					\item
						People do not notice, get habituated
				\end{itemize}
			\item
				Training / education: What is feasible / effective?
		\end{itemize}
	\item
		Awareness Requirements on Users: Be suspicious, even if you know the sender, the message fits your current situation and fits your work and life practices

		Be suspicious of everything!
	\item
		Decision making by Kahneman: two ways System 1 (intuition, effortless) and System 2 (rational reasoning)
	\item
		Can we make users aware by catching them on insecure behavior? Phishing as a service?
		\begin{itemize}
			\item
				2015 Police division in Berlin
			\item
				Email: Store all your work and private passwords in the secure password storage of the Berlin police. Corporate Design, Signature from non-existing person
			\item
				Half of subjects clicked, 7.5\% gave their credentials (35 individuals)
			\item
				Technical solution: visually distinguish internal \& external emails to help the users, save effort and avoid mistakes
		\end{itemize}
	\item
		Why is (Spear) phishing awareness difficult?
		\begin{itemize}
			\item
				Security aware behavior is difficult to maintain (constant vigilance, system 2) is tiring and unhealthy
			\item
				Life and work practices clash with expected security behavior: people receive lots of attachments, how can they distinguish legitimate ones? Lots of (wasted?) time, when checking every link received
			\item
				Social norms clash with expected security behavior (\enquote{what would they think if I [insert security behavior]}) See door policy
				
		\end{itemize}
	\item
		Countermeasures: What do you want people to \emph{do}? Will they be \emph{able} to do this? Effects on Producitvity, well-being, trust relationships. Evidence needed!
	\item
		Should we train users at all? Is the defense worth the effort?
	\item
		\fullcite{herley2009so}
		\begin{itemize}
			\item
				Fully rational individual user should reject most of the security advice existing today. Focus on individual users, not businesses
			\item
				Cost-benefict calculations:
				\begin{itemize}
					\item
						Protect from direct costs: attacks --- small or unknown costs
					\item
						Burden with indirect costs (externalities) through following security advice --- larce costs (time, nerves, money)
				\end{itemize}
			\item
				US phsihing annual costs of \$60 million
			\item
				US online population of 180 million users
			\item
				33 cents per person per year
			\item
				With minimum wage, anti-phishing education should be 2.6 minutes annually
		\end{itemize}
	\item
		And in Germany?
		\begin{itemize}
			\item
				Phishing Cost 2014 (peak): 27.9 million EUR
			\item
				Online Banking Users: 37 million people
			\item
				75 cents per year
			\item
				5.3 minutes per year
		\end{itemize}
	\item
		Pishing Risk = probability of attack * cost of attack
		\begin{itemize}
			\item
				Probability of being phished: 7000 cases in 2014, $37\times10^6$ customers $\Rightarrow$ 0.0002
			\item
				Expected cost of being phished: 28 Mio / 7000 cases = 4000 EUR per case
			\item
				Expected effort for getting money back if phished: 1 day (8 h), 1 week (40 h), 5 weeks (200h)?

				Average wage is 20 EUR/h, even with 5 weeks only 4000 Euro
			\item
				Risk of being phished = probability of attack * cost of attack $\Rightarrow 0.0002 \times4000 = 0.8$ EUR
			\item
				Only applicable for individual users, not businesses
		\end{itemize}
	\item
		RSA breach (2011)
		\begin{itemize}
			\item
				Started with spear phishing emails to small groups of employees (not high profile or high value target)
			\item
				One employee in HR received mail out of their junk folder (to make attack detection more difficult), double-clicked on the email attachement (because fits job expectation, curiosity)
			\item
				Consequences: SecureID maybe hacked, very popular 2FA tool, 40 millions tokens changed

				70 Million USD spent on aftermath of attack  
			\item
				Perimeter security (don't let attackers in!) failed, but in-depth defenses worked
		\end{itemize}
	\item
		Feasibility of Anti-Phishing Training
		\begin{itemize}
			\item
				Perimeter defense against spear phishing is difficult (exploit indispensable human characteristics, heuristics, work and life practices)
			\item
				False negatives: How to tell a fraud attempt from legitimate requests?
		\end{itemize}
	\item
		Effectiveness of Anti-Phishing Training: 50\%$\rightarrow$75--85\% with students.

		No evidence of effectiveness in real-world business environment
\end{itemize}
\subsection{Scam}
\begin{itemize}
	\item
		\fullcite{herley2012nigerian}
		\begin{itemize}
			\item
				Somebody who believes this, can be made to believe anything; somebody who does not fall for this is not worth the effort, at some point the are likely to notice that this is a scam
		\end{itemize}
	\item
		\fullcite{lea2009psychology}
		\begin{itemize}
			\item
				Study about scams in the UK, Interview with Scam Victims, Text Mining of scams, Scam Simulation
			\item
				Features of Scam Victims
				\begin{itemize}
					\item
						Better than average knowledge and experience in the scam (prize draws, lottery, investments) area 
					\item
						Lack of emotional control

					\item
						More effort in cognitive analysis of Scam (\enquote{Is this a good offer? Should I? Why? Why not?})
					\item
						Long-odds gamble: little to loose, a lot to win? (Misunderstanding: cannot win)
					\item
						Hiding response to scam, even from their more rational selves
				\end{itemize}
			\item
				How Scam works:
				\begin{itemize}
					\item
						Appeals to trust and authority
					\item
						Visceral triggers (personalization, big money, urgency --- provoke intuitive reactions, reduce thinking)
					\item
						Scarcity cues (urgency and uniqueness)
					\item
						Behavioral commitment (small steps of compliance --- sunken cost fallacy?)
					\item
						Disproportion between prize and cost/effort --- phantom fixation: concentrate on the huge prize
				\end{itemize}
			\item
				Conclusions:
				\begin{itemize}
					\item
						Scam victims have distinct psychological features, but not bad decision makers generally
					\item
						10--20\% of UK population (estimated)
				\end{itemize}
		\end{itemize}
	\item
		Psychology of Persuasion:
		\begin{itemize}
			\item
				How to influence people to do or think something (in IT security: Scam (make people exhibit bad behavior) or Security Mechanisms / Advice (Make them exhibit good behavior) )

			\item
				6 principles of Influence:
				\begin{enumerate}
					\item
						Reciprocity (Erwiderung) --- I've done you good, so now you do me good
					\item
						Commitment and Consistency --- Once you made a choice or decided to do something, you usually proceed with doing this
					\item
						Social Proof --- if everybody around you is doing this, you will do this too (canned laughter in TV, looking in the same direction as everybody else)
					\item
						Authority --- just do/think what experts tell you do do / to think
					\item
						Liking --- Tupperware parties, famous athlete in ads, good cop / bad cop
					\item
						Scarcity --- It's urgent! Do it now, or you will lose opportunity / money
				\end{enumerate}
		\end{itemize}
	\item
		\enquote{The Real Hustle}

		7 Principles:
		\begin{enumerate}
			\item
				Distraction (Primary interest gets the whole attention, Nigerian Scam)
			\item
				Social Compliance (Follow authority or what other people are doing)
			\item
				Herd (If other people are doing that, then it's okay) (ex: Hütchenspiel and reputation systems, e.g. on amazon
			\item
				Dishonesty (if the victim can made to behave badly, then they are not likely to admit this fact to police, \dots (Nigerian Scams) )
			\item
				Kindness (People are generally nice and ready to help)
			\item
				Need and Greed (if somebody wants something very much, they disregard the risk)
			\item
				Time (Decisions under pressure are often suboptimal or wrong)
		\end{enumerate}

		In Exam, only need to explain the 6 or the 7 principles, not both
\end{itemize}

\subsection{Social Engineering}
\begin{itemize}
	\item
		The art of Deception, by Mitnick, former hacker and social engineer
	\item
		Against Security: about security at airports, subways, cities. But can be applied to IT security. You only see security, but completly disregard other goals people have.
	\item
		Defense against Social Engineering
		\begin{itemize}
			\item
				\enquote{Pretexting}: use an invented scenario to obtain information

				Example: private detective wants to find out health status of person under observation. Call doctor or insurance of person, claiming to be a doctor from another hospital, having the POI under their care in coma

				One insurance in the UK, about 30 fake requests per week
		\end{itemize}
	\item
		Trust bias (most people are trustworthy) vs deception bias (most people are deceicive) (see System 1/2)
	\item
		What would be the consequences of constant vigilance?
\end{itemize}


\section{Usable security}
\begin{itemize}
	\item
		Approaches to usable security
		\begin{itemize}
			\item
				Invisible security
				\begin{itemize}
					\item
						Automated filtering not good enough
					\item
						Other technical measures also no 100\% protection
					\item
						Automated security may confuse users
				\end{itemize}
			\item
				Understandable security:
				\begin{itemize}
					\item
						Can we help people recognize possible dangerous situations?
					\item
						WARNINGS: warn users about possible dangers and let them decide what to do
				\end{itemize}
			\item
				Train the users
				\begin{itemize}
					\item
						Phishing: user effort doesn not match benefit
					\item
						Training might be infeasible or have negative effects
					\item
						User training and education is difficult also in other security areas
				\end{itemize}
		\end{itemize}

	\item
		What does User awareness/training/education mean?

		Sometimes it means Perception (I know there is a threat), sometimes Protection (I know threats and know solutions) and Behavior (I also act right)

		Systematisation attempt by Benenson: Knowledge about threats (Awareness?), Knowledge about protection mechanisms (education?) and Behavior (Training?)
	\item
		Fear Messages / \enquote{How do Consumers react to Cybercrime?}
		\begin{itemize}
			\item
				Fear deters from online banking more than actual identity theft
		\end{itemize}
	\item
		\fullcite{stewart2012death}
		\begin{itemize}
			\item
				Extended parallel process model
			\item
				See slides
		\end{itemize}
	\item
		\fullcite{nielsen2004user}
		\begin{itemize}
			\item
				Security is too complicated for users --- user education does not work
			\item
				User education puts burden on the wrong shoulders
			\item
				User education may scare people --- deter them from realizing full benefits of technology
		\end{itemize}
	\item
		Alternatives to User Education
		\begin{itemize}
			\item
				Differences between physical and virtual world $\Rightarrow$ Criminals in a virtual world have more possibilities, Users have less possibilities to notice attacks
			\item
				Technology changes to help users:
				\begin{enumerate}
					\item
						Security Settings on by default
					\item
						Automate updates
					\item
						Provide usable security features
					\item
						Hunt down criminals
				\end{enumerate}
		\end{itemize}
	\item
		Automatic updates are problematic, because sometimes users think they have automatic updates while actually they don't
	\item
		Examples of invisible security?

		Antivirusprogramme, SSL/TLS, Sandboxing, Typsicherheit von Programmiers
	\item
		\fullcite{edwards2008security}
		\begin{itemize}
			\item
				Fixed policy: hard-coded in the system
			\item
				Customizable policy: set by experts or set by users (Firewalls, passwords, cookies)
			\item
				Dynamic policy: Spam filter
			\item
				Sometimes security measures are not appropriate (e.g. encrypted emails)
			\item
				Sometimes it's not polite to block \enquote{untrusted} websites
			\item
				Sometimes concerns and values of users and developers do not align (example passwords, )
		\end{itemize}
\end{itemize}

\section{Example exam questions}
\subsection{Passwords}
\begin{itemize}
	\item
		Factual questions
		\begin{itemize}
			\item
				How would you define a \enquote{strong} password?
			\item
				Discuss in which cases requiring passwords of pre-defined \enquote{strength} makes sense, and in which cases it does not make sense
			\item
				What are security and usability problems of password usage? How can they be mitigated? (at least, partially)
			\item
				Which factors should be taken into account when an organization defines a password policy?
		\end{itemize}
	\item
		Transfer questions
		\begin{itemize}
			\item
				What is a secure and usable password?
			\item
				What be your personal advice on the handling of passwords to a home computer user? What are advantages and disadvantages of your advice?
			\item
				What does it mean for an authentication mechanism to be usable? To be secure?
			\item
				Which alternative authentication mechanisms do you know? How do they compare to passwords in terms of security and usability?
			\item
				How would you nvestigate security and usability of an authentication mechanism?
		\end{itemize}
\end{itemize}

\subsection{Usable Security}
\begin{itemize}
	\item
		How would you define usability?
	\item
		How should usability definition be adapted when reasoning about usability of security?
	\item
		What does it mean for a security mechanism or policy to be usable?
	\item
		What happens if a secure system has bad usability? Examples?
	\item
		How can usability of security mechanisms be tested?
	\item
		What is the relation between security and usability? In which cases do they constitute a trade-off? How can they be aligned with each other?
	\item
		Which three approaches do you know making security usable? Discuss advantages and disadvantages, give examples
	\item
		What other system properties, apart from usability should be satisfied in order for the system to be accepted
	\item
		You are commisioned to develop a security mechanism X / security policy Y. How would you organize the development process?
\end{itemize}

\subsection{Cyberfraud}
\begin{itemize}
	\item
		Why is phishing an important security threat? What difficulties people have in recognizing phish?
	\item
		How to determine which measures against phishing are economically justifiable? Which kind of data you need in order to answer the question? (no need for statistics from the lecture)
	\item
		Why do Nigerian scammers say they are from Nigeria?
	\item
		What psychological tricks make people commit errors of judgement and fall for scam in consequence?
	\item
		Explain principles of psychological influence with security-related examples
	\item
		(Transfer) Is user education a good measure against phishing or social engineering for home computer users / for employees? Argue your opinion!
\end{itemize}

\end{document}
