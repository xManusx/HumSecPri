\section{Example exam questions}
\subsection{Passwords}
\begin{itemize}
	\item
		Factual questions
		\begin{itemize}
			\item
				How would you define a \enquote{strong} password?
			\item
				Discuss in which cases requiring passwords of pre-defined \enquote{strength} makes sense, and in which cases it does not make sense
			\item
				What are security and usability problems of password usage? How can they be mitigated? (at least, partially)
			\item
				Which factors should be taken into account when an organization defines a password policy?
		\end{itemize}
	\item
		Transfer questions
		\begin{itemize}
			\item
				What is a secure and usable password?
			\item
				What be your personal advice on the handling of passwords to a home computer user? What are advantages and disadvantages of your advice?
			\item
				What does it mean for an authentication mechanism to be usable? To be secure?
			\item
				Which alternative authentication mechanisms do you know? How do they compare to passwords in terms of security and usability?
			\item
				How would you nvestigate security and usability of an authentication mechanism?
		\end{itemize}
\end{itemize}

\subsection{Usable Security}
\begin{itemize}
	\item
		How would you define usability?
	\item
		How should usability definition be adapted when reasoning about usability of security?
	\item
		What does it mean for a security mechanism or policy to be usable?
	\item
		What happens if a secure system has bad usability? Examples?
	\item
		How can usability of security mechanisms be tested?
	\item
		What is the relation between security and usability? In which cases do they constitute a trade-off? How can they be aligned with each other?
	\item
		Which three approaches do you know making security usable? Discuss advantages and disadvantages, give examples
	\item
		What other system properties, apart from usability should be satisfied in order for the system to be accepted
	\item
		You are commisioned to develop a security mechanism X / security policy Y. How would you organize the development process?
\end{itemize}
