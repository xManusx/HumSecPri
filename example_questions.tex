\section{Example exam questions}
\subsection{Passwords}
\begin{itemize}
	\item
		Factual questions
		\begin{itemize}
			\item
				How would you define a \enquote{strong} password?
			\item
				Discuss in which cases requiring passwords of pre-defined \enquote{strength} makes sense, and in which cases it does not make sense
			\item
				What are security and usability problems of password usage? How can they be mitigated? (at least, partially)
			\item
				Which factors should be taken into account when an organization defines a password policy?
		\end{itemize}
	\item
		Transfer questions
		\begin{itemize}
			\item
				What is a secure and usable password?
			\item
				What be your personal advice on the handling of passwords to a home computer user? What are advantages and disadvantages of your advice?
			\item
				What does it mean for an authentication mechanism to be usable? To be secure?
			\item
				Which alternative authentication mechanisms do you know? How do they compare to passwords in terms of security and usability?
			\item
				How would you nvestigate security and usability of an authentication mechanism?
		\end{itemize}
\end{itemize}

\subsection{Usable Security}
\begin{itemize}
	\item
		How would you define usability?

		Inwiefern ein Produkt genutzt werden kann, um ein bestimmtes Ziel zu erreichen, im Hinblick auf Effektivität, Effizienz und Zufriedenheit in einem genau spezifizierten Kontext
	\item
		How should usability definition be adapted when reasoning about usability of security?

		Erstes Johnny paper

		Security software ist usable, wenn die Personen, die die Software benutzen sollen:
		\begin{enumerate}
			\item
				zuverlässig auf Security tasks aufmerksam gemacht werden
			\item
				herausfinden können, wie sie diese Tasks erfolgreich bewältigen können
			\item
				keine gefährlichen Fehler machen (können)
			\item
				mit dem Interface zufrieden genug sind, um es weiter zu benutzen
		\end{enumerate}
	\item
		What does it mean for a security mechanism or policy to be usable?

		Ein mechanismus oder eine Policy ist usable

		\begin{itemize}
			\item
				The extent to which the mechanism or policy can be used by
			\item
				Specified users
			\item
				Um ein bestimmtes security goal (manche davon unsichtbar oder secondary für den user)
			\item
				und das primary goal zu erreichen
			\item
				Mit Effektivität, Effizienz und Zufriedenheit
			\item
				Während der Ausführung der spezifizierten primary tasks
			\item
				in einem spezifizierten context
			\item
				unter Einbezug von spezifizierten Attacken
		\end{itemize}
	\item
		What happens if a secure system has bad usability? Examples?

		user finden work-arounds, benutzen unsichere Systeme
	\item
		How can usability of security mechanisms be tested?
		
		User study with role playing
	\item
		What is the relation between security and usability? In which cases do they constitute a trade-off? How can they be aligned with each other?

		Security verbietet, usability erlaubt

	  Usability erlaubt gute Dinge, Security verbietet böse Dinge

		Im Besten Fall unsichtbare Security, funktioniert einfach!
	\item
		Which three approaches do you know making security usable? Discuss advantages and disadvantages, give examples

		Unsichtbare Security, funktioniert einfach, technische Limits durch schlechte Accuracy, manchmal auch nicht sinnvolle Sachen tun wollen

		Understandable security: sichtbar und intuitiv machen. Schwierig unbedarften Nutzern die Hintergründe näher zu bringen, passende Metaphern/Analogien zu finden

		User education: Funktioniert nicht so gut, kostet Zeit und Geld (lohnt sich oft nicht $\Rightarrow$ phishing education!), User muss Dinge verstehen,  User muss motiviert sein!
	\item
		What other system properties, apart from usability should be satisfied in order for the system to be accepted
	\item
		You are commisioned to develop a security mechanism X / security policy Y. How would you organize the development process?

		Identify S\&P Goals and attacks, measures, effective?, efficient?

		Sinnvollerweise security schon von Anfang an der Entwicklung Teil des Produktes. Oft schwierig Security später mit einzubauen
\end{itemize}

\subsection{Cyberfraud}
\begin{itemize}
	\item
		Why is phishing an important security threat? What difficulties people have in recognizing phish?

		Passwörter können abhanden kommen, schwierig automatisch zu identifizieren, kann über verschiedene Kanäle kommen.
		User haben falsche Entscheidungs strategien um festszustellen was Phishing ist: (Email für mich, von normaler Firma, die machen das so), (Neugier, Plausibler Content), schauen oft nur auf Content der Website, nicht auf Security Indicators
	\item
		How to determine which measures against phishing are economically justifiable? Which kind of data you need in order to answer the question? (no need for statistics from the lecture)

		Jährliche Kosten durch Phishing / Potentiell betroffene Personen (Internet User) = Kosten / Person / Jahr

		Kann man über Mindestlohn in Zeit umrechnen, ist nicht so viel
	\item
		Why do Nigerian scammers say they are from Nigeria?

		Jemand der so eine doofe Geschichte glaubt, kann quasi dazu gebracht werden alles zu glauben. \enquote{Intelligentere} Leute fallen da schon nicht drauf rein, sind also keine leichten Ziele
	\item
		What psychological tricks make people commit errors of judgement and fall for scam in consequence?

		An Vertrauen und Autorität appelieren, große Gewinne, künstliche Knappheit (nur jetzt, sonst zu spät!), commitment (immer nur kleine Schritte machen)
	\item
		Explain principles of psychological influence with security-related examples
		\begin{itemize}
			\item
				Erwiderung: Phishing email (wir haben das für dich getan, jetzt musst du das hier noch bestätigen\dots und dein Passwort eingeben)
			\item
				Commitment: Scam emails
			\item
				Social proof: In Phishing Email: \enquote{5000 Personen haben die neuen AGBs schon bestätigt\dots und ihr Passwort eingegeben}
			\item
				Authority: Warnungen mit Angstmachen?
			\item
				Liking: 
			\item
				Scarcity: In Scam Emails (schnell machen!)

		\end{itemize}

	\item
		(Transfer) Is user education a good measure against phishing or social engineering for home computer users / for employees? Argue your opinion!

		For home users: effort doesn't match benefit, training might be infeasible or have negative effects, erzeugt Angst (bringt Leute mehr davon ab Online Banking zu benutzen als tatsächlicher Identity Theft!)
\end{itemize}

\subsection{User education, Awareness, training}
\begin{itemize}
	\item
		Awareness, education, training: what are the differences between these terms?

		Awareness = Wissen um Angriffe, education = Wissen um Verteidignungsmaßnahmen, Training = Richtiges Verhalten
	\item
		What are pros and cons of security measures that try to change the users by means of awareness, education and training?

		Schwierig für User, zu schwierig. Gibt Verantwortung an die falschen (devs und admins sollten für Sicherheit verantworlich sein!). Macht Leuten Angst (Online banking, id theft vs. fear mongering)
	\item
		In which cases is a fear-inducing security message a good idea? (Explain the extended parallel processing model, awareness poster for identity theft)

		Stimulus kommt an (nachricht, warnung)

		First appraisal: perceived threat = entweder weiter zu 2nd appraisal oder kein Threat wahrgenommen

		Second appraisal: Perceived efficay, dann wenn Wirksame Gegenmaßnahme möglich protection motivation und adaptive changes, falls keine wirksamen Gegenmaßnahmen möglich/bekannt dann defensive motivation und maladaptive changes
	\item
		Which factors should be considered when designing a user education campaign?
\end{itemize}

\subsection{Warnings}
\begin{itemize}
	\item
		Compare and constrast passive and active warnings

		Passive: Indikatoren, sehr ineffektiv, oft gar nicht bemerkt

		Aktive: Unterbrechen User, oft auch ineffektiv, Automatische Antwort (einfach wegklicken), oft nicht verstanden
	\item
		What makes a good/bad warning or indicator? Examples?
	\item
		Which factors influence responses to warnings?
	\item
		Which issues arise when security experts want people to comply with security warnings?
	\item
		In which cases can non-compliance with a warning be a better decision than compliance?
\end{itemize}

\subsection{Privacy}
\begin{itemize}
	\item
		How would you define Privacy? What is the difference between privacy and confidentiality?
	\item
		How do the views on privacy differ between EU and USA?
	\item
		What is the Westin privacy index used for? What problems exist with the usage of this index?
	\item
		Explain the meaning of privacy paradox / privacy dichotomy
	\item
		Explain the rational choice theory in economics and how decision making  in privacy deviates from it
	\item
		Explain and give examples of
		\begin{itemize}
			\item
				Asymmetric information
			\item
				Bounded rationality
			\item
				Behavioral biases
		\end{itemize}
	\item
		Give examples on how the decisions to share or to withhold sensitive information are dependent on the context in which the request for information is made
		\begin{itemize}
			\item
				Seriousness
			\item
				Endowment effect / order effect
		\end{itemize}
	\item
		Why is privacy considered to be a human right? What are the serious consequences for individuals and society if privacy is violated?
		\begin{itemize}
			\item
				Feature creep, aggregation, exclusion
			\item
				Problems with the \enquote{I've got nothing to hide} argument
		\end{itemize}
	\item
		What are arguments for and against data retention?
	\item
		Is NSA-type surveillance a good measure against crime and terrorism? Argue why or why not, (depending on you opinion) and give examples
	\item
		What role does risk overestimation play in the societal and political debate about data retention / NSA surveillance? Explain how and why risk overestimation happens and what are the consequences of overreaction.
\end{itemize}
